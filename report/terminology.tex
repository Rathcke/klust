\section{Terminology}
This section describes some basic terminology and notation of this text.

The words \emph{sequence} and \emph{string} will be used to denote the same
concepts, i.e. an ordered list of objects, where an object will most often be a
text character. In general, a sequence or string might be infinite, but in this
project they will always be finite unless mentioned otherwise.

In this text, the notion of a subsequence is different from that of a
substring, as is the common convention: a \emph{substring} of a sequence $S$ is
a consecutive, ordered list of objects, that occurs in $S$, while a
\emph{subsequence} of $S$ is a sequence that can be obtained from $S$ by
deleting some objects from the sequence without changing the order of the
objects.

We will sometimes say that a sequence \emph{matches} another sequence, or that
it matches a centroid, to denote that the two sequences are within the given
distance threshold, or greater than or equal to the given similarity threshold.

The terms \emph{query sequence} and \emph{target sequence} will be used to
denote the sequence being searched for and a sequence being compared to,
respectively.

The name \texttt{USEARCH} denotes both the piece of software developed by
Robert C. Edgar and the algorithm used in the program for searching for
sequences in some database, e.g. for searching for a centroid that is within
the given distance threshold of some sequence. The name \texttt{UCLUST} refers
to the clustering algorithm used in the \texttt{USEARCH} program.


\subsection{Notation}
Let $s$ and $t$ be sequences.
\begin{itemize}
  \item $|s|$ denotes the length of $s$
  \item $s \sqsubseteq t$ denotes that $s$ is a substring of $t$
  \item $J(A,B)$ denotes the Jaccard index, or Jaccard similarity coefficient,
    of sets $A$ and $B$
\end{itemize}
