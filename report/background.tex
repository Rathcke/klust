\section{Background}

\subsection{Distance metrics}

\subsubsection{Edit distance}
One type of \emph{edit distance} is the \emph{Levenshtein} distance, which is a
string metric for determining the similarity between two sequences. It is
defined to be the minimum number of edits to turn the first sequence into the
other. \\
The edit operations consists of \emph{insertions}, \emph{deletions} and
\emph{substitutions}. These operations are, respectively, inserting a letter,
removing a letter and changing one letter for another. \\
The Levenshtein distance from sequence $a$ to $b$ with indicator function 
$1_(a_i\neq b_j)$, which is only $0$ when $a_i=b_j$, can mathematically be 
defined as
\begin{equation}
	\operatorname{lev}_{a,b}(i,j)= 
	\begin{cases}
	  	\max(i,j) & \text{ if} \min(i,j)=0, \\
  	  	\min \begin{cases}
        		\operatorname{lev}_{a,b}(i-1,j) + 1 \\
          	\operatorname{lev}_{a,b}(i,j-1) + 1 \\
          	\operatorname{lev}_{a,b}(i-1,j-1) + 1_{(a_i \neq b_j)}
       \end{cases} & \text{ otherwise.}
	\end{cases}
\end{equation}
where $(i,j)=(|a|,|b|)$.\\
For example, the two sequences
\begin{center}
\texttt{ACGT} \\
\texttt{ACGGC}
\end{center}
would have a distance of 2 (substituting T for G and inserting a C). \\
However, there are some cases where the relevance of the distance is arguable.
Consider the sequences
\begin{center}
\texttt{AACC} \\
\texttt{CCAA}
\end{center}
with a distance of 4. The two sequences actually have the maximal distance
possible even though they have a close resemblance to each other.


\subsubsection{Feature based distance}
A $k$-mer, or $k$-gram or simply a \emph{word}, is a sequence of length
$k \geq 0$ over some alphabet $\mathcal{A}$ of a sequence. $k$-mers are a type
of sequence \emph{feature}. An interesting feature of a sequence is the
$k$-mers that occur in that sequence.

$d2$ is a feature based distance metric, using $k$-mers as the feature. The
distance is calculated by counting the $k$-mers occurring in two sequences,
representing these occurrences as two vectors and finally taking the Euclidean
distance between these two vectors.

Let $c_x(w)$ be the number of times that a $k$-mer $w$ occurs in the sequence
$x$. Then the $d2$ distance can be defined as follows:
\begin{equation}
  d2_k(x,y) \eqdef \sqrt{\sum_{|w|=k} (c_x(w) - c_y(w))^2}
\end{equation}

As an example, the two $2$-mer frequency vectors of the sequences
\begin{align*}
  S_1 &= AGACTG \\
  S_2 &= ACAGAT
\end{align*}
over the alphabet $\mathcal{A} = \{A,C,T,G\}$, can be illustrated as follows:

\begin{table}[!h]
\centering
\scalebox{0.7}{
\begin{tabular}{c | c c c c c c c c c c c c c c c c}
        & AA & AC & AG & AT & CA & CC & CG & CT & GA & GC & GG & GT & TA & TC & TG & TT \\
  \hline
  $S_1$ &    &  1 &  1 &    &    &    &    &  1 &  1 &    &    &    &    &    &  1 &    \\
  \hline
  $S_1$ &    &  1 &  1 &  1 &  1 &    &    &    &  1 &    &    &    &    &    &    &    \\
\end{tabular}}
\end{table}

The Euclidean distance would then be calculated as
\begin{align*}
  d2_2(S_1, S_2)
    &= \sqrt{(1-1)^2 + (1-1)^2 + (-1)^2 + (-1)^2 + (1)^2 + (1-1)^2 + (1)^2} \\
    &= 2
\end{align*}

To better support distance between two sequences of different length, but
similar subsequences, the concept of a \emph{window} can be used to split the
distance calculation into comparison of substrings of length of the window
and then use the window with the least distance as the resulting distance. In
this way two sequences, where one is a prefix or postfix of the other, will
have zero distance.




\subsubsection{Sequence alignment}
\emph{Sequence alignment} is used in bioinformatics to identify regions of
similarity by arranging two or more sequences in a certain manner. Besides
shifting a sequence to a side, gaps can be inserted between objects as well. We
represent the gap with '-' and it indicates a insertion in the sequence or a
deletion from another sequence.\\
We call a column that includes - for an indel. If it does not and all object in
the column are same, then we have a match. Otherwise, it is a mismatch.\\
\begin{figure}[h!]
  \centering
  	  \begin{align*}
  	  	\mbox{\texttt{ATGCAACGA}} \\
      	\mbox{\texttt{-TGCG-CGA}}
  	  \end{align*}
  \caption{Sequence alignment of 'ATGCAACGA' and 'TGCGCGA'}
  \label{fig:seqAlignment}
\end{figure}
Figure \ref{fig:seqAlignment} displays an alignment of two sequences. In
column 1 is an indel, in column 2 are matches and in column 5 is a mismatch. 
The amount of each type can then be used in a formula to calculate the 
similarity between the sequences. Note that more than two sequences can be 
aligned in a \emph{multiple
sequence alignment}. \\
There are many alignments. Among those is an optimal alignment which is the one that is searched for. \\
\\
Both \texttt{VSEARCH} and \texttt{UCLUST} uses some kind of sequence alignment. \texttt{VSEARCH} uses a parallelized version of the dynamic programming algorithm Needleman-Wunsch, while \texttt{UCLUST} uses an heuristic procedure by default, but can be modified to use Needleman-Wunsch or Smith-Waterson. However, this will make \texttt{UCLUST} much slower.

\subsection{Cluster analysis algorithms}

\subsubsection{Hierarchical clustering}
There are typically two types of hierarchical algorithms, both of which are 
merge based, the agglomerative that works in a bottom-up manner and the 
divisive that has a top-down approach, but we focus on explaining the 
agglomerative. The agglomerative works in a bottom-
up manner where all sequences start as leaves and they are then merged 
together in clusters if they are within a given threshold.\\
\\% TODO: MAKE ILLUSTRATION
This adresses the question of what is required for a cluster to be within a 
given threshold. Besides measuring similarity between sequences with a 
distance metric, a definition for when a sequence is part of a cluster is 
needed. \\

With the \textit{single-linkage} approach, only one sequence from both 
clusters are required to be within the threshold.

In the \textit{complete-linkage} approach, all sequences from one cluster 
must be within the threshold of all sequences from the other.

In the \textit{average-linkage}, the average distance of all pairs of 
sequences is required to be within the threshold.\\
\\
The singly-linkage approach is often used in sequence clustering though it has its disadvantages as argued in \cite[pp. 62-63]{dong}

\subsubsection{Graph based clustering}


\subsubsection{Greedy algorithm like in \texttt{UCLUST}} 
The \texttt{UCLUST} algorithm is greedy and is developed so that all member
sequence have similarity $\geq$ $T$ to their centroid.  It works by processing
one sequence at a time and comparing these to the existing centroids. If a
match is found, the sequence is assigned to the matching centroid, otherwise it
becomes a centroid of a new cluster.

It is designed to also support that each centroid has a similarity $<T$ to all
other centroids, but this is not guaranteed to hold. The reason is that the
algorithm only compares a sequence to a prespecified number of centroids given
by the \texttt{-maxrejects} parameter.

It uses the $k$-mer counts to locate centroid candidates for a match, though
how it does it is not very well described.

The similarity calculations are performed with global sequence alignments as
\cite{usearch} claims that using the word counts to calculate identity is not
sensitive enough.

%TODO? \subsubsection{Greedy algorithm with recalculation of centroids}
