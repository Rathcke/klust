\section{Research process}
\subsection{Starting out: Levenshtein with dynamic programming}
For a start, a naive version of the Levenshtein distance metric was
implemented, but as expected, this is absolutely useless in any practical
setting, since it does not reuse already calculated results. Therefore, this
was quickly transformed into a dynamic programming solution, which solves
subproblems just once, stores and reuses the intermediate results.

\subsection{$d2$ distance}
After learning that the basic Levenshtein algorithm is far too slow for the
problem domain of this project, we turned our attention to the $d2$ distance
metric. The first edition of the $d2$ distance metric algorithm is shown in
figure \ref{alg:d2_naive}. This algorithm maintains a single frequence vector,
as a map structure to allow for large $k$ which results in a large number of
possible different $k$-mers. The map is indexed using the lexicographical
position of the $k$-mer. When iterating through the first sequence, $k$-mer
counts are incremented and when iterating through the second sequence, they are
decremented. Finally all the frequensies are squared and the square root of the
sum is returned.


\begin{algorithm}
  \caption{Naive \textsc{d2} distance metric}
  \label{alg:d2_naive}
  \begin{algorithmic}[1]
    \Require{$s$ and $t$ are DNA or RNA sequences and $k \in \mathbb{Z}^+$}
    \Statex
    \Function{d2}{$s, t, k$}
      \State initialize \texttt{freq\_map}
      \For{$i \gets 0$ to $length(s) - k$}
        \State freq\_map[index\_of\_kmer(s.substring(i, k)]\texttt{++}
      \EndFor
      \For{$i \gets 0$ to $length(t) - k$}
        \State freq\_map[index\_of\_kmer(t.substring(i, k)]\texttt{--}
      \EndFor
      \State $total \gets 0$
      \ForAll{$e \in$ freq\_map}
        \State $total \gets e.value^2$  \Comment{calculate the Euclidean distance}
      \EndFor
      \State \Return $\sqrt{total}$
    \EndFunction
  \end{algorithmic}
\end{algorithm}
