\documentclass[12pt,a4paper]{article}

\usepackage[dot, autosize, outputdir="dotgraphs/"]{dot2texi}
\usepackage{tikz}
\usetikzlibrary{shapes}
\usepackage[utf8]{inputenc}
\usepackage{amsmath}
\usepackage{mathtools}
\usepackage{amsfonts}
\usepackage{lastpage}
\usepackage{pdfpages}
\usepackage{gauss}
\usepackage{fancyvrb}
\usepackage{graphicx}
\usepackage{url}

\usepackage{fancyhdr, lastpage}
\pagestyle{fancy}
\fancyhf{}
\renewcommand{\headrulewidth}{0pt}
\cfoot{Page \thepage\ of \pageref{LastPage}}

\title{Bachelor project: Synopsis \\ \vspace{2mm}
      {\LARGE DNA/RNA-sequence clustering algorithm}}
\author{Anders Kiel Hovgaard \and Nikolaj Dybdahl Rathcke}
\date{February 23, 2015}

\begin{document}
\maketitle
\thispagestyle{fancy}

\section{Problem statement}
\begin{quotation}
  Can we implement an algorithm that can cluster DNA/RNA-sequencing data, of
  sizes at least half a million sequences of 50-500 characters each, and
  compete with the performance of \texttt{UCLUST}? If not, how close can we
  get?
\end{quotation}

% Can we implement an algorithm that can cluster sequencing data and compete
% with the performance of 'uclust'? And if not, how close can we get?"

% En præcisering af hvilken størrelse data vi taler om, måske et minimum?
% En afgrænsning omkring at vi ikke nødvendigvis laver implementation eller
% algoritme fra scratch.
% En afgrænsning om hvorvidt vi vil analysere det formelt.
% En afgrænsning om hvorvidt vi vil afprøve programmet på de rigtige data.

\subsection{Restrictions on the scope of the project}
Clustering of protein sequences will not be considered in this project, i.e.
only RNA and DNA sequences will be considered.


\section{Motivation}
The problem consists of clustering huge amounts of DNA/RNA-sequences (up to 500
million strings of 50-500 characters each), based on a given similarity
threshold, so that any one sequence only appears in exactly one cluster.
 
There are not many available algorithms and tools for efficient clustering of
sequencing data. The one tool
\texttt{UCLUST}\footnote{\url{http://drive5.com/usearch/}} that does the job is
closed-source and cost money.

The goal of this project is to research the possibilities of creating an open
tool that can match the performance of the proprietary version of
\texttt{UCLUST}.

%Is it possible to create a tool that can match the performance of the
%proprietary version of uclust?

\end{document}
