\documentclass[11pt,a4paper]{article}

\usepackage[dot, autosize, outputdir="dotgraphs/"]{dot2texi}
\usepackage{tikz}
\usetikzlibrary{shapes}
\usepackage[utf8]{inputenc}
\usepackage{amsmath}
\usepackage{mathtools}
\usepackage{amsfonts}
\usepackage{pdfpages}
\usepackage{gauss}
\usepackage{fancyvrb}
\usepackage{graphicx}
\usepackage{url}
\usepackage{float}

% headers and footers
\usepackage{fancyhdr, lastpage}
\pagestyle{fancy}
\fancyhf{}
\renewcommand{\headrulewidth}{0pt}
\cfoot{Page \thepage\ of \pageref{LastPage}}

% Gantt diagrams
\usepackage{gantt}

% Alternative titles:
%   "Efficient clustering of massive sequencing data"
%   "Efficient DNA/RNA-sequence clustering"
%   "Efficient clustering of massive amounts of DNA/RNA-sequences"
\title{Bachelor project: Synopsis \\
       \vspace{2mm}
       {\LARGE Efficient DNA/RNA-sequence clustering}} % TODO: without "algorithm" ?
\author{Anders Kiel Hovgaard \and Nikolaj Dybdahl Rathcke}
\date{February 23, 2015}

\begin{document}
\maketitle
\thispagestyle{fancy}

\section{Problem statement}
Can we implement an algorithm that can cluster DNA/RNA-sequencing data, of
sizes at least half a million sequences of 50-500 characters each, and compete
with the performance of \texttt{UCLUST}? If not, how close can we get?

% Can we implement an algorithm that can cluster sequencing data and compete
% with the performance of 'uclust'? And if not, how close can we get?"

% En præcisering af hvilken størrelse data vi taler om, måske et minimum?
% En afgrænsning omkring at vi ikke nødvendigvis laver implementation eller
% algoritme fra scratch.
% En afgrænsning om hvorvidt vi vil analysere det formelt.
% En afgrænsning om hvorvidt vi vil afprøve programmet på de rigtige data.

\subsection{Restrictions on the scope of the project}
%The algorithm to be implemented might either be an original algorithm that we
%create or a modified version of an existing algorithm. Similarly for the
%implementation, we might reuse parts of existing open implementation to some
%extent.
%The same holds for the implementation: we might build upon and change an
%existing implementation, we might start from scratch or we might just reuse
%parts of existing open implementations. % TODO: too weak?

The algorithm to be implemented might either be one that we design from scratch
or a modified version of an existing algorithm. Similarly for the
implementation, we might reuse parts of existing open implementation to some
extent.

Clustering of protein sequences will not be considered in this project, i.e.
only RNA and DNA sequences will be considered.


\section{Motivation}
The main motivation for this project comes from a need for efficient tools for
clustering of sequence data, and related techniques, in the microbiology
department at the University of Copenhagen. In particular, the idea for this
project comes from a collaboration between the supervisor of this project,
Rasmus Fonseca, and Martin Asser Hansen, who is a bioinformatician in the
Molecular Microbial Ecology
Group\footnote{\url{http://www2.bio.ku.dk/microbiology/}}.

The problem consists of clustering huge amounts of DNA/RNA-sequences (up to 500
million strings of 50-500 characters each), based on a given similarity
threshold, so that any one sequence only appears in exactly one cluster.
 
There are not many available algorithms and tools for efficient clustering of
sequencing data. The one tool
\texttt{UCLUST}\footnote{\url{http://drive5.com/usearch/}} that does the job is
closed-source and considered too expensive. % and cost money.

The goal of this project is to research the possibilities of creating an open
tool that can match the performance of the proprietary version of
\texttt{UCLUST}.


\section{Tasks and time planning}
The following tasks are not necessarily dependent on each other in the order
presented, for instance, the implementation of the distance metric and
clustering algorithm will be interleaved with testing and optimization.
Thus, some of the tasks will be completed in parallel. Furthermore, the report
will be written concurrently with some of the tasks.

A Gantt diagram for the project plan is shown in figure \ref{fig:gantt}.

\begin{itemize}
  \item Survey existing literature, algorithms and implementations of sequence
    clustering. ($\sim$1-2 weeks)

  \item Choose distance metric and parameters, design algorithm for calculating
    distance. ($\sim$1 week)

  \item Design initial version of clustering algorithm. ($\sim$1-2 weeks)

  \item Implement distance metric. ($\sim$1-2 weeks)

  \item Implement clustering algorithm. ($\sim$3-4 weeks)

  \item Analyze and profile the implementation, investigate possibilities for
    optimization and perform these on the implementation. Possibly perform
    parallelization of the implementation. ($\sim$3-4 weeks)

  \item Perform ongoing testing after implementation of initial algorithm and
    in-between optimizations and modifications. ($\sim$3-4 weeks)

  \item Analyze the time and space complexity of the algorithm. ($\sim$1-2
    weeks)
\end{itemize}

\begin{figure}[H]
  \begin{gantt}[xunitlength=0.5cm,fontsize=\small,titlefontsize=\small]{10}{20}
    \begin{ganttitle}
      \titleelement{Feb}{4}
      \titleelement{Mar}{4}
      \titleelement{Apr}{4}
      \titleelement{May}{4}
      \titleelement{Jun}{4}
    \end{ganttitle}
    \ganttbar{Alg. design}{2}{4}
    \ganttbar{Impl. of metric}{4}{10}
    \ganttbar{Impl. of alg.}{4}{10}
    \ganttbar{Optimization}{8}{8}
    \ganttbar{Testing}{5}{11}
    \ganttbar{Complexity}{12}{4}
    \ganttbar{Writing}{3}{14}
    \ganttmilestone[color=cyan]{Report}{17}
    \ganttmilestone[color=red]{Defense}{18}
  \end{gantt}
  \caption{Gantt diagram of the project plan.}
  \label{fig:gantt}
\end{figure}


\section{Product}
We will design and implement an algorithm for efficient clustering of massive
amounts of DNA/RNA-sequences, most likely in a programming language such as
\texttt{C} or \texttt{C++}. Furthermore, we will write a report that documents
our efforts, the algorithms and implementations produced and evaluates their
performance.

\newpage
\subsection{Alternative task list}
\textbf{Initial algorithm design}
\begin{itemize}
  \item Product: \textit{An initial design of an algorithm that performs
    clustering on DNA/RNA sequences. Choice of distance metric and parameters.}
  \item Dependencies: \textit{None.}
  \item Workload: \textit{10 workdays}
\end{itemize}

\noindent
\textbf{Implementation of distance metric}
\begin{itemize}
  \item Product: \textit{An implementation of the chosen distance metric in
    \texttt{C}/\texttt{C++} or similar.}
  \item Dependencies: \textit{Distance metric from "Initial algorithm design".
    Ongoing testing and optimization.}
  \item Workload: \textit{10 workdays}
\end{itemize}

\noindent
\textbf{Algorithm implementation}
\begin{itemize}
  \item Product: \textit{An implementation of the algorithm in
    \texttt{C}/\texttt{C++} or similar.}
  \item Dependencies: \textit{Clustering algorithm from "Initial algorithm
    design" and the product from the task "Implementation of distance metric".
    Ongoing testing and optimization}
  \item Workload: \textit{20 workdays}
\end{itemize}

\noindent
\textbf{Optimization}
\begin{itemize}
  \item Product: \textit{A modified algorithm with reduced running time.}
  \item Dependencies: \textit{Ongoing process with the tasks "Implementation of
    distance metric", "Algorithm implementation" and "Testing".}
  \item Workload: \textit{20 workdays}
\end{itemize}

\noindent
\textbf{Testing}
\begin{itemize}
  \item Product: \textit{Ongoing feedback on the algorithm and optimization.}
  \item Resources: \textit{Real life data and a powerful computer with similar
    specifications to what would be used.}
  \item Dependencies: \textit{Ongoing process with the tasks "Implementation of
    distance metric", "Algorithm implementation" and "Optimization"}
  \item Workload: \textit{10 workdays}
\end{itemize}

\noindent
\textbf{Time complexity analysis}
\begin{itemize}
  \item Product: \textit{An analysis of the algorithm and possibly major
    optimizations.}
  \item Dependencies: \textit{Algorithm and major optimizations.}
  \item Workload: \textit{5 workdays}
\end{itemize}


\end{document}
